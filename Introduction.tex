\section{INTRODUCTION}
The intention of this experiment is to characterize the radiation pattern of an LED empirically and verify it analytically. To do this, we make a few critical assumptions.
\begin{enumerate}
\item We treat the LED as uniform planar light source.
\item The radiation pattern of the LED is independent of angle $\phi$.
\item The experiment is performed at a scale such that the we can treat the LED as a point source and the aperture area as differential.
\end{enumerate}

We mounted an LED on a rotating platform and recorded power perceived by a power detector at chosen distance R from the LED for angles between -90$\degree$ and 90$\degree$. Using this data, we are able to determine the total power emitted by the LED and compare it with our analytically derived expected results. We refined our model in an attempt to account for numerical discrepancies between the expected and observed results. Unfortunately, as we were in a hurry, we forgot to record our station number and battery voltage. However, examination of the rigorous procedure used to setup the apparatus leads us to conclude that any variation in between our results and the correct answer would be due to variation in LEDs. Support for this conclusion is offered in the Discussion section.