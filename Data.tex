\section{DATA}
\subsection{Collection Methodology}
The following precautions were taken to minimize any error introduced in collecting data.

\begin{itemize}
    \item We set the aperture radius precisely by closing it flush around a nominally \SI{.7}{mm} stick of lead. We later measured the section of the lead stick that we inserted in the aperture with a caliper and found it was actually \SI{.69}{mm} in diameter.
    \item We re-purposed multimeter boxes - which we used as four black walls surrounding our apparatus, minimizing variations in light noise which we noticed could be attributed to movements of the observer. The resulting change in millivolts which corresponds to power detected after installation of the black walls was notable, having decreased by about \SI{20}{\milli\volt}. Inevitable background radiation was still present, but having audited previous measurements at certain angles at random, and finding no difference, we safely conclude the background radiation was effectively constant, which we account for in our subsequent computations.
    \item We once again used an ordinary pencil to precisely determine the radius of the hemisphere, that is, the distance from the the approximate center of the LED where light is emitted, to the sensor plate within the power detector. We place the pencil's eraser's tip on the curved tip of the LED and the other end of the pencil towards the aperture of the power detector. We then clicked the pencil so that the pencil's lead extended past the opening of the aperture just \SI{5}{mm} in front of the sensor. We then used a caliper to measure the length of the extended pencil and added \SI{4}{mm} back to account for the radius of the LED, and another \SI{2}{mm} for the distance from the lead tip to the sensor plate.
\end{itemize}

\newpage
\subsection{Collected Data}
The raw collected voltage measurements can be found in \autoref{app:raw data} The most important parameters that we measured are listed below in \autoref{tab:meas_params}.


\begin{table}[!htb]
\caption{Measured Parameters} \label{tab:meas_params}
\bigskip
\centering
\begin{tabular}{l|l}
\rowcolor[HTML]{C0C0C0} 
\textbf{Parameter} & \textbf{Value} \\ \hline
Aperture Diameter & \SI{0.69}{mm} \\
Aperture Area & \SI{0.374}{mm^2} \\
Distance from LED to Sensor ("R") & \SI{153.63}{mm} \\
Maximum Recorded Voltage & \SI{527.7}{mV} \\
Maximum Recorded Power & \SI{1.0352}{\micro\watt} \\
Resistor Voltage & \SI{6.54}{V} \\
Battery Voltage & \SI{9.00}{V} \\
Bias Resistance & \SI{1.182}{\kilo\ohm} \\
Ambient Noise & \SI{5.1}{\milli\volt}
\end{tabular}
\end{table}

\begin{figure}[H]
    \centering
\begin{tikzpicture}
\begin{axis}[
    xticklabels={%
      -$\frac{\pi}{2}$,
      -$\frac{\pi}{4}$,
      0,
      $\frac{\pi}{4}$,
      $\frac{\pi}{2}$,
    },
    xmin=-1.5707963267948966,xmax=1.5707963267948966,
    ymin=0,ymax=1,
    xtick={-1.5707963267948966,-0.7853981633974483,...,1.5707963267948966},
    ytick={0,0.2,...,1},
    x coord trafo/.code=\pgfmathparse{#1*360/(2*pi)},
    title=,
    ylabel={Watts},
    legend entries={normalized,cosine}
]
\addplot[no marks,blue] table[col sep=comma, x={Angle}, y={wNorm}]
{dataPlot.csv};
\addplot[no markers,red] table[col sep=comma, x={Angle}, y={BestFitCosine}]
{dataPlot.csv};
\end{axis}
\end{tikzpicture}
\caption{Observed Power (Normalized) vs. Predicted Power (Lambertian Cosine)}
\end{figure}
