\newpage
\section{DISCUSSION}
Our results reveal two modes within our radiation pattern. Cursory research revealed a point source is a poor model for a domed LED. The ideal shape for an LED semiconductor die is a microsphere. Current manufacturing limitations however, only allow semiconductor die that sits within the post-anvil to re reliably cut into a cuboid. The typical manufacturing processes for LED’s yield dies with radiation patterns that are best characterized by 6 light cones emerging from a center in an octahedral fashion. The emitted power plot(Figure 1) reveals two modes - what actually appears to be the sum of two raised cosines centered left and right of zero degrees. An easy explanation for this might be two emission cones that are mirrored across the on-axis. This would result in the observed dip at 0 degrees. We propose a new model, the sum of two cosines shifted left and right of center to peak at the two maximums, to account for the two light cones. Using a least squares fit yielded
\[\cos^{127}(\theta-.1396) + \cos^{127}(\theta+.1396)\]

\begin{figure}[H]
    \centering
\begin{tikzpicture}
\begin{axis}[
    xticklabels={%
      -$\frac{\pi}{2}$,
      -$\frac{\pi}{4}$,
      0,
      $\frac{\pi}{4}$,
      $\frac{\pi}{2}$,
    },
    xmin=-1.5707963267948966,xmax=1.5707963267948966,
    ymin=0,ymax=1,
    xtick={-1.5707963267948966,-0.7853981633974483,...,1.5707963267948966},
    ytick={0,0.2,...,1},
    x coord trafo/.code=\pgfmathparse{#1*360/(2*pi)},
    title=,
    ylabel={Watts},
    legend entries={Normalized Observed Power, $\cos^{n}$},
    legend style={anchor=west,xshift=-2cm},
]
\addplot[no markers,blue] table[col sep=comma, x={Angle}, y={wNorm}]
{dataPlot.csv};
\addplot[no markers,red] table[col sep=comma, x={Angle}, y={Re-Fitted Cosine}]
{dataPlot.csv};
\end{axis}
\end{tikzpicture}
\caption{Observed Power (Normalized) vs. Predicted Power (Two Cone Model)}
\end{figure}

A possible reason for the discrepancy between the observed and analytically derived total power may be an improper model. Using this new model, the error diminishes to 1.42 times the observed power. Another possible reason for this discrepancy are the two rippling nodes at $\frac{-3\pi}{4}$ and $\frac{3\pi}{4}$. These become greatly amplified when multiplied by $\sin(\theta)$ prior to integration. The raised cosine function however does not have such large ripples at those location. We are fairly certain those ripples are not a product of measurement error. As we mentioned earlier, we audited values (we had previously measured) at randomly chosen angles and found little to no differences. We also reliably eliminated variation in noise due to the environment. 
\newline 
Cursory research seems to suggest that the octahedral cone light source model of an LED is itself lacking. We also eliminated the LED's physical installation above the platform as a source of error. At first glance, the center of the LED appears to be tracing out an arc as the platform is rotated, which would result in a changing distance for the LED to the power detector. Closer examination reveals however, that the plane supporting the LED is actually recessed a aligning the LED's center nearly perfectly with the platform's axis of rotation. In light of all the care taken to ensure error free observation, we can only suggest that the raised cosine model is an incomplete model for a domed LED(at this distance away from the power detector). Such a conclusion nullifies the assumptions presented in the introduction.