\documentclass[10pt,a4paper]{article}
\usepackage{microtype}
\usepackage{setspace}
\usepackage{textcomp}
\usepackage{gensymb}
\usepackage[hidelinks]{hyperref}
\newcommand*{\Appendixautorefname}{Appendix}
\usepackage[letterpaper, portrait, margin=1in]{geometry}
\usepackage[utf8]{inputenc}
\usepackage{amsmath}
\usepackage{amsfonts}
\usepackage{amssymb}
\usepackage[table,xcdraw]{xcolor}
\usepackage{pgfplotstable,filecontents,booktabs}\usepackage{pgfplots}
\usepackage[toc,page,title]{appendix}
\usepackage{minted}
\usepackage{siunitx}
\usepgfplotslibrary{polar}
\pgfplotsset{compat=1.13}


\title{RADIATION PATTERN OF A LIGHT-EMITTING DIODE}
\author{Yehowshua Immanuel \and Noah Daugherty}

\begin{document}
\maketitle
\tableofcontents
\newpage

\section{INTRODUCTION}
The intention of this experiment is to characterize the radiation pattern of an LED empirically and verify it analytically. To do this, we make a few critical assumptions.
\begin{enumerate}
\item We treat the LED as uniform planar light source.
\item The radiation pattern of the LED is independent of angle $\phi$.
\item The experiment is performed at a scale such that the we can treat the LED as a point source and the aperture area as differential.
\end{enumerate}

We mounted an LED on a rotating platform and recorded power perceived by a power detector at chosen distance R from the LED for angles between -90$\degree$ and 90$\degree$. Using this data, we are able to determine the total power emitted by the LED and compare it with our analytically derived expected results. We refined our model in an attempt to account for numerical discrepancies between the expected and observed results. Unfortunately, as we were in a hurry, we forgot to record our station number and battery voltage. However, examination of the rigorous procedure used to setup the apparatus leads us to conclude that any variation in between our results and the correct answer would be due to variation in LEDs. Support for this conclusion is offered in the Analysis section.
\section{DATA}
\subsection{Collection Methodology}
The following precautions were taken to minimize any error introduced in collecting data.

\begin{itemize}
    \item We set the aperture radius precisely by closing it flush around a nominally \SI{.7}{mm} stick of lead. We later measured the section of the lead stick that we inserted in the aperture with a caliper and found it was actually \SI{.69}{mm} in diameter.
    \item We re-purposed multimeter boxes - which we used as four black walls surrounding our apparatus, minimizing variations in light noise which we noticed could be attributed to movements of the observer. The resulting change in millivolts which corresponds to power detected after installation of the black walls was notable, having decreased by about \SI{20}{\milli\volt}. Inevitable background radiation was still present, but having audited previous measurements at certain angles at random, and finding no difference, we safely conclude the background radiation was effectively constant, which we account for in our subsequent computations.
    \item We once again used an ordinary pencil to precisely determine the radius of the hemisphere, that is, the distance from the the approximate center of the LED where light is emitted, to the sensor plate within the power detector. We place the pencil's eraser's tip on the curved tip of the LED and the other end of the pencil towards the aperture of the power detector. We then clicked the pencil so that the pencil's lead extended past the opening of the aperture just \SI{5}{mm} in front of the sensor. We then used a caliper to measure the length of the extended pencil and added \SI{4}{mm} back to account for the radius of the LED, and another \SI{2}{mm} for the distance from the lead tip to the sensor plate.
\end{itemize}

\newpage
\subsection{Collected Data}
The raw collected voltage measurements can be found in \autoref{app:raw data} The most important parameters that we measured are listed below in \autoref{tab:meas_params}.


\begin{table}[!htb]
\caption{Measured Parameters} \label{tab:meas_params}
\bigskip
\centering
\begin{tabular}{l|l}
\rowcolor[HTML]{C0C0C0} 
\textbf{Parameter} & \textbf{Value} \\ \hline
Aperture Diameter & \SI{0.69}{mm} \\
Aperture Area & \SI{0.374}{mm^2} \\
Distance from LED to Sensor ("R") & \SI{153.63}{mm} \\
Maximum Recorded Voltage & \SI{527.7}{mV} \\
Maximum Recorded Power & \SI{1.0352}{\micro\watt} \\
Resistor Voltage & \SI{6.54}{V} \\
Battery Voltage & \SI{9.00}{V} \\
Bias Resistance & \SI{1.182}{\kilo\ohm} \\
Ambient Noise & \SI{5.1}{\milli\volt}
\end{tabular}
\end{table}

\begin{figure}[bp]
    \centering
\begin{tikzpicture}
\begin{axis}[
    xticklabels={%
      -$\frac{\pi}{2}$,
      -$\frac{\pi}{4}$,
      0,
      $\frac{\pi}{4}$,
      $\frac{\pi}{2}$,
    },
    xmin=-1.5707963267948966,xmax=1.5707963267948966,
    ymin=0,ymax=1,
    xtick={-1.5707963267948966,-0.7853981633974483,...,1.5707963267948966},
    ytick={0,0.2,...,1},
    x coord trafo/.code=\pgfmathparse{#1*360/(2*pi)},
    title=,
    ylabel={Watts},
    legend entries={normalized,cosine}
]
\addplot[no markers,blue] table[col sep=comma, x={Angle}, y={wNorm}]
{dataPlot.csv};
\addplot[no markers,red] table[col sep=comma, x={Angle}, y={BestFitCosine}]
{dataPlot.csv};
\end{axis}
\end{tikzpicture}
\caption{Observed Power (Normalized) vs. Predicted Power (Lambertian Cosine)}
\end{figure}

\section{RESULTS}

\subsection{Normalized Voltages}
A table containing Normalized Voltages can be found in \autoref{app:norm data}. To normalize our data, we subtracted the ambient voltage (\SI{5.1}{\milli\volt}) from all of our measured values and then divided by the maximum voltage observed (\SI{517.6}{\milli\volt} at $\theta = -14\degree$).
    %a) A table of normalized voltages, derived from the data table in the second section
%%%%%%%%%%%%%%%%%%%%%%%%%%%%%%%%%%%%%%%%%%%% 
\subsection{Raised Cosine Parameter}
The value of the Raised Cosine Parameter $n$ was found using the MATLAB Code in \autoref{app:matlab code}. We normalized our data and fit it to a raised cosine using a least squares fit. The formula is provided below for reference:
\[\text{error} = \frac{1}{N}\sum_{i=1}^{N}[V_{norm}(\theta_{si}) - \cos^{n}(\theta_{si})]^{2}\]
We used a range of exponents ranging from 0 to 100 with a 0.1 step size and found the best fit exponent to be 33.1 with a normalized error of 0.0077.

    %b) The value of the raised cosine parameter, n, determined by the curve-fitting procedure. Details on the curve fit process should be provided as necessary.
%%%%%%%%%%%%%%%%%%%%%%%%%%%%%%%%%%%%%%%%%%%%   
\subsection{Peak Radiation Intensity}
To find the approximate peak radiation intensity, we used the equation:
\[K_{0}\approx \frac{P_{0}}{A_{r}}R^{2}\: \: [W/sr] = .0165 [W/sr]\]
which uses the maximum power $P_0$, determined from our measurements to be \SI{.258}{\micro\watt}, the aperture area measured as \SI{0.374}{mm^2}, and the distance R from the LED to the aperture of \SI{153.63}{mm}.  
    %c) The value of the peak radiation intensity, K0 in the proper units
%%%%%%%%%%%%%%%%%%%%%%%%%%%%%%%%%%%%%%%%%%%%
\subsection{Total Emitted Power}
We numerically computed the power emitted by the LED in the hemisphere facing the apparatus using the equation provided below.
\[\int_{0}^{2\pi}{\int_{0}^{\frac{\pi}{2}}K(\theta)\cdot \sin(\theta) d \theta d\phi}\]
We obtained an observed total power of $\SI{4.7}{mW}$.

\subsection{Supplied Electrical Power and Wall Plug Efficiency}
We analytically computed the total power emitted using the value for $K_0$.
\[P = \frac{2 \pi K_{0}}{n+1}\: \: [W] =  \frac{2 \pi .0165} {33+1}\: \: [W] = \SI{.03}{W}\]
To calculate the power supplied to the LED, we measured the voltage across a resistor of know value. The resulting power consumed by the LED is $(9V - 6.54V)*(.0005A) = \SI{13}{mW}$. 
Thus, the output efficiency of our LED is $\frac{3}{13} = .23$. Various sources in academia confirm that a domed LED has a wall-plug efficiency of 0.26, so our value of .23 is not unreasonable.

    %d) The calculated total emitted power
    %e) The calculated supplied electrical power and the wall-plug efficiency
%%%%%%%%%%%%%%%%%%%%%%%%%%%%%%%%%%%%%%%%%%%% 
\subsection{Extra Credit}
As mentioned before, we computed the total power emitted numerically using trapezoid integration which yielded a value of $\SI{4.7}{mW}$. We also analytically computed a value of $ \SI{.03}{W}$. The total power emitted we obtained is 1.5 times larger than our analytically derived value. Some reasone for this discrepancy are presented in the discussion.
\subsection{Resultant Data}

\begin{table}[!htb]
\caption{Calculated Results} \label{tab:calc_results}
\bigskip
\centering
\begin{tabular}{l|l}
\rowcolor[HTML]{C0C0C0} 
\textbf{Parameter} & \textbf{Value} \\ \hline
Raised Cosine Exponent, $n$ & 33.1 \\
Peak Radiation Intensity, $K_0$ & \SI{66.2}{mW/sr} \\
Total Emitted Power, $P_e$ & \SI{16.7}{mW} \\
Supplied Electrical Power, $P_s$ & \SI{12.2}{mW} \\
Wall-Plug Efficiency & 63.235\%
\end{tabular}
\end{table}

%%%%%%%%%%%%%%%%%%%%%%%%%%%%%%%%%%%%%%%%%%%%
\subsection{Polar Plots}
\begin{figure}[H]
    \centering
    \begin{tikzpicture}
    \begin{polaraxis}[
        xticklabels={%
          -$\frac{\pi}{2}$,
          -$\frac{\pi}{4}$,
          0,
          $\frac{\pi}{4}$,
          $\frac{\pi}{2}$,
        },
        xmin=-1.5707963267948966,xmax=1.5707963267948966,
        ymin=0,ymax=1,
        xtick={-1.5707963267948966,-0.7853981633974483,...,1.5707963267948966},
        ytick={0,0.2,...,1},
        ytick pos=left,
        yticklabel style={anchor=east},
        ylabel style={anchor=near ticklabel,rotate=90},
        x coord trafo/.code=\pgfmathparse{#1*360/(2*pi)},
        title= {},
        ylabel={Watts},
        legend entries={Observed Power(W)},
        legend style={anchor=west}
    ]
    \addplot[only marks,blue] table[col sep=comma, x={Angle}, y={wNorm}]
    {dataPlot.csv};
    {dataPlot.csv};
    \end{polaraxis}
    \end{tikzpicture}
    \caption{Observed Power (Normalized)}
    \label{fig:polar power}
\end{figure}

\begin{figure}[H]
    \centering
    \begin{tikzpicture}
    \begin{polaraxis}[
        xticklabels={%
          -$\frac{\pi}{2}$,
          -$\frac{\pi}{4}$,
          0,
          $\frac{\pi}{4}$,
          $\frac{\pi}{2}$,
        },
        xmin=-1.5707963267948966,xmax=1.5707963267948966,
        ymin=0,ymax=1,
        xtick={-1.5707963267948966,-0.7853981633974483,...,1.5707963267948966},
        ytick={0,0.2,...,1},
        ytick pos=left,
        yticklabel style={anchor=east},
        ylabel style={anchor=near ticklabel,rotate=90},
        x coord trafo/.code=\pgfmathparse{#1*360/(2*pi)},
        title= {},
        ylabel={Watts},
        legend entries={Observed Power(W), Fitted Power(W) $\cos ^n$},
        legend style={anchor=west}
    ]
    \addplot[only marks,blue] table[col sep=comma, x={Angle}, y={wNorm}]
    {dataPlot.csv};
    \addplot[no markers,red] table[col sep=comma, x={Angle}, y={BestFitCosine}]
    {dataPlot.csv};
    \end{polaraxis}
    \end{tikzpicture}
    \caption{Observed Power (Normalized) vs. Predicted Power (Lambertian Cosine)}
    \label{fig:polar power}
\end{figure}






    %f) A polar plot (in the manner of Figs. 2 and 3), showing all data points at angle location (plotted as discrete legible points, please)
    %g) A polar plot, over-layed on the polar plot of the data points, showing the raised cosine function (a solid curve) obtained from your curve fit.
    % Polar Plot


\newpage
\section{DISCUSSION}
Our results reveal two modes within our radiation pattern. Cursory research revealed a point source is a poor model for a domed LED. The semiconductor die that sits within the post-anvil structure inside the LED must be cut with saw cuts into a cuboid. The typical manufacturing processes for LED’s yield dies with radiation patterns that are best characterized by 6 light cones emerging from a center in an octahedral fashion. The emitted power plot(Figure 1) reveals two modes - what actually appears to be the sum of two raised cosines centered left and right of zero degrees. An easy explanation for this might be two emission cones that are mirrored across the on-axis. This would result in the observed dip at 0 degrees. We propose a new model, the sum of two cosines shifted left and right of center to peak at the two maximums, to account for the two light cones. Using a least squares fit yielded
\[\cos^{127}(\theta-.1396) + \cos^{127}(\theta+.1396)\]

\begin{figure}[H]
    \centering
\begin{tikzpicture}
\begin{axis}[
    xticklabels={%
      -$\frac{\pi}{2}$,
      -$\frac{\pi}{4}$,
      0,
      $\frac{\pi}{4}$,
      $\frac{\pi}{2}$,
    },
    xmin=-1.5707963267948966,xmax=1.5707963267948966,
    ymin=0,ymax=1,
    xtick={-1.5707963267948966,-0.7853981633974483,...,1.5707963267948966},
    ytick={0,0.2,...,1},
    x coord trafo/.code=\pgfmathparse{#1*360/(2*pi)},
    title=,
    ylabel={Watts},
    legend style={at={(axis cs:0.5,1)},anchor=south west}
]
\addplot[no markers,blue] table[col sep=comma, x={Angle}, y={wNorm}]
{dataPlot.csv};
\addplot[no markers,red] table[col sep=comma, x={Angle}, y={Re-Fitted Cosine}]
\legend{Normalized Observed Power, $\cos^{n}$}
{dataPlot.csv};
\end{axis}
\end{tikzpicture}
\caption{Observed Power (Normalized) vs. \newline Predicted Power (Two Cone Model)}
\end{figure}

%%Appedices
\newpage
\begin{appendices}
\section{Collected Raw Data}
\label{app:raw data}
\pgfplotstabletypeset[col sep=comma,sci zerofill,fixed,fixed zerofill,showpos,
    column type=r,
    every head row/.style={
        before row={\toprule\multicolumn{2}{c}{Angle} & \multicolumn{1}{c}{Voltage (mV)} & \multicolumn{2}{c}{Angle} & \multicolumn{1}{c}{Voltage (mV)}\\},
        after row=\midrule},
    every last row/.style={after row=\bottomrule},
    display columns/0/.style={select equal part entry of={0}{2},string type},
    display columns/1/.style={select equal part entry of={0}{2},string type},
    display columns/2/.style={select equal part entry of={0}{2},string type,column type=r||},
    display columns/3/.style={select equal part entry of={1}{2},string type},
    display columns/4/.style={select equal part entry of={1}{2},string type},
    display columns/5/.style={select equal part entry of={1}{2},string type},
    columns/Meas. Angle/.style={precision=0,column name=Measured},
    columns/Adj. Angle/.style={precision=0,column name=Adjusted},
    columns/Meas. Voltage (mV)/.style={precision=1,column name=Measured},
    columns={Meas. Angle,Adj. Angle,Meas. Voltage (mV),Meas. Angle,Adj. Angle,Meas. Voltage (mV)},
    ]{data.csv}
    \newpage

\newpage
\section{Normalized Voltages}
\label{app:norm data}
%%% TABLE START: Normalized Data
\pgfplotstabletypeset[col sep=comma,sci zerofill,fixed,fixed zerofill,showpos,
    column type=r,
    every head row/.style={
        before row={\toprule\multicolumn{2}{c}{Angle} & \multicolumn{1}{c}{Voltage (mV)} & \multicolumn{2}{c}{Angle} & \multicolumn{1}{c}{Voltage (mV)}\\},
        after row=\midrule},
    every last row/.style={after row=\bottomrule},
    display columns/0/.style={select equal part entry of={0}{2},string type},
    display columns/1/.style={select equal part entry of={0}{2},string type},
    display columns/2/.style={select equal part entry of={0}{2},string type,column type=r||},
    display columns/3/.style={select equal part entry of={1}{2},string type},
    display columns/4/.style={select equal part entry of={1}{2},string type},
    display columns/5/.style={select equal part entry of={1}{2},string type},
    columns/Meas. Angle/.style={precision=0,column name=Measured},
    columns/Adj. Angle/.style={precision=0,column name=Adjusted},
    columns/Norm. Voltage (mV)/.style={precision=4,column name=Normalized},
    columns={Meas. Angle,Adj. Angle,Norm. Voltage (mV),Meas. Angle,Adj. Angle,Norm. Voltage (mV)},
    ]{data.csv}

\end{appendices}

\end{document}
%\begin{align}
%\end{align}
