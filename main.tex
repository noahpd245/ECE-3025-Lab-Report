\documentclass[10pt,a4paper]{article}
\usepackage{setspace}
\usepackage{gensymb}
\usepackage[hidelinks]{hyperref}
\usepackage[letterpaper, portrait, margin=1in]{geometry}
\usepackage[utf8]{inputenc}
\usepackage{amsmath}
\usepackage{amsfonts}
\usepackage{amssymb}
\usepackage[table,xcdraw]{xcolor}

\title{RADIATION PATTERN OF A LIGHT-EMITTING DIODE}
\author{Yehowshua Immanuel, Noah Daugherty}

\begin{document}
\maketitle
\tableofcontents
\newpage

\section{INTRODUCTION}
The intention of this experiment is to characterize the radiation pattern of an LED empirically and verify it analytically. To do this, we make a few critical assumptions.
\begin{enumerate}
\item We treat the LED as a uniform planar light source.
\item The radiation pattern of the LED is independent of angle $\phi$.
\item The experiment is performed at a scale such that the we can treat the LED as a point source and the aperture area as differential.
\end{enumerate}

We mounted an LED on a rotating platform and recorded power perceived by a power detector at chosen distance R from the LED for angles between -90$\degree$ and 90$\degree$. Using this data, we are able to determine the total power emitted by the LED and compare it with our analytically derived expected results. We refined our model in an attempt to account for numerical discrepancies between the expected and observed results. Unfortunately, as we were in a hurry, we forgot to record our station number and battery voltage. However, examination of the rigorous procedure used to setup the apparatus leads us to conclude that any variation in between our results and the correct answer would be due to variation in LEDs. Support for this conclusion is offered in the Discussion section.

\section{DATA}
\subsection{Collection Methodology}
The following precautions were taken to minimize any error introduced in collecting data.

\begin{itemize}
    \item We set the aperture radius precisely by closing it flush around a nominally .7mm stick of lead.
    \item We re-purposed multimeter boxes - which we used as four black walls surrounding our apparatus, minimizing variations in light noise which we noticed could be attributed to movements of the observer. The resulting change in milliVolts detected after installation of the black walls was notable, having decreased by about 20mV. Inevitable background radiation was still present, but having audited previous measurements at certain angles at random, and finding no difference, we safely conclude the background radiation was effectively constant, which we account for in our subsequent computations.
\end{itemize}

\subsection{Tabulated Results}
The raw collected data can be found in Appendix A

%%Appedices
\newpage
\clearpage
\begin{appendices}
\section{Collected Raw Data}
\label{app:raw data}
\begin{table}[h]
\centering
\pgfplotstabletypeset[col sep=comma,sci zerofill,fixed,fixed zerofill,showpos,
    column type=r,
    every head row/.style={
        before row={\toprule\multicolumn{2}{c}{Angle} & \multicolumn{1}{c}{Voltage (\si{mV})} & \multicolumn{2}{c}{Angle} & \multicolumn{1}{c}{Voltage (\si{mV})}\\},
        after row=\midrule},
    every last row/.style={after row=\bottomrule},
    display columns/0/.style={select equal part entry of={0}{2},string type},
    display columns/1/.style={select equal part entry of={0}{2},string type},
    display columns/2/.style={select equal part entry of={0}{2},string type,column type=r||},
    display columns/3/.style={select equal part entry of={1}{2},string type},
    display columns/4/.style={select equal part entry of={1}{2},string type},
    display columns/5/.style={select equal part entry of={1}{2},string type},
    columns/Meas. Angle/.style={precision=0,column name=Measured},
    columns/Adj. Angle/.style={precision=0,column name=Adjusted},
    columns/Meas. Voltage (mV)/.style={precision=1,column name=Measured},
    columns={Meas. Angle,Adj. Angle,Meas. Voltage (mV),Meas. Angle,Adj. Angle,Meas. Voltage (mV)},
    ]{data.csv}
\end{table}

\clearpage

\section{Normalized Voltages}
\label{app:norm data}
%%% TABLE START: Normalized Data
\begin{table}[h]
\centering
\pgfplotstabletypeset[col sep=comma,sci zerofill,fixed,fixed zerofill,showpos,
    column type=r,
    every head row/.style={
        before row={\toprule\multicolumn{2}{c}{Angle} & \multicolumn{1}{c}{Voltage (\si{mV})} & \multicolumn{2}{c}{Angle} & \multicolumn{1}{c}{Voltage (\si{mV})}\\},
        after row=\midrule},
    every last row/.style={after row=\bottomrule},
    display columns/0/.style={select equal part entry of={0}{2},string type},
    display columns/1/.style={select equal part entry of={0}{2},string type},
    display columns/2/.style={select equal part entry of={0}{2},string type,column type=r||},
    display columns/3/.style={select equal part entry of={1}{2},string type},
    display columns/4/.style={select equal part entry of={1}{2},string type},
    display columns/5/.style={select equal part entry of={1}{2},string type},
    columns/Meas. Angle/.style={precision=0,column name=Measured},
    columns/Adj. Angle/.style={precision=0,column name=Adjusted},
    columns/Norm. Voltage (mV)/.style={precision=4,column name=Normalized},
    columns={Meas. Angle,Adj. Angle,Norm. Voltage (mV),Meas. Angle,Adj. Angle,Norm. Voltage (mV)},
    ]{data.csv}
\end{table}

\clearpage
\section{Matlab Code}
\label{app:matlab code}
\inputminted{Matlab}{Matlab.m}

\end{appendices}

\end{document}
%\begin{align}
%\end{align}

