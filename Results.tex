\section{RESULTS}

\subsection{Normalized Voltages}
A table containing Normalized Voltages can be found in \autoref{app:norm data}. To normalize our data, we subtracted the ambient voltage (\SI{5.1}{\milli\volt}) from all of our measured values and then divided by the maximum voltage observed (\SI{517.6}{\milli\volt} at $\theta = -14\degree$).
    %a) A table of normalized voltages, derived from the data table in the second section
%%%%%%%%%%%%%%%%%%%%%%%%%%%%%%%%%%%%%%%%%%%% 
\subsection{Raised Cosine Parameter}
The value of the Raised Cosine Parameter $n$ was found using the MATLAB Code in \autoref{app:matlab code}. We normalized our data and fit it to a raised cosine using a least squares fit. The formula is provided below for reference:
\[\text{error} = \frac{1}{N}\sum_{i=1}^{N}[V_{norm}(\theta_{si}) - \cos^{n}(\theta_{si})]^{2}\]
We used a range of exponents ranging from 0 to 100 with a 0.1 step size and found the best fit exponent to be 33.1 with a normalized error of 0.0077.

    %b) The value of the raised cosine parameter, n, determined by the curve-fitting procedure. Details on the curve fit process should be provided as necessary.
%%%%%%%%%%%%%%%%%%%%%%%%%%%%%%%%%%%%%%%%%%%%   
\subsection{Peak Radiation Intensity}
To find the approximate peak radiation intensity, we used the equation:
\[K_{0}\approx \frac{P_{0}}{A_{r}}R^{2}\: \: [W/sr]\]
which uses the maximum power $P_0$, determined from our measurements to be \SI{1.0352}{\micro\watt}, a computed aperture area of \SI{0.374}{mm^2}, and distance R from the LED to the aperture of \SI{153.63}{mm}. 
    %c) The value of the peak radiation intensity, K0 in the proper units
%%%%%%%%%%%%%%%%%%%%%%%%%%%%%%%%%%%%%%%%%%%%
\subsection{Total Emitted Power}
We numerically computed the power emitted by the LED in the hemisphere facing the apparatus using the equation provided below.
\[\int_{0}^{2\pi}{\int_{0}^{\frac{\pi}{2}}K(\theta)\cdot \sin(\theta) d \theta d\phi}\]
We obtained an observed total power of $\SI{4.7}{mW}$.

\subsection{Supplied Electrical Power and Wall Plug Efficiency}
We analytically computed the total power emitted using the value for $K_0$.
\[P = \frac{2 \pi K_{0}}{n+1}\: \: [W] =  \frac{2 \pi .0165} {33+1}\: \: [W] = \SI{.03}{W}\]
To calculate the power supplied to the LED, we measured the voltage across a resistor of know value. The resulting power consumed by the LED is $(9V - 6.54V)*(.0005A) = \SI{13}{mW}$. 
Thus, the output efficiency of our LED is $\frac{3}{13} = .23$. Various sources in academia confirm that a domed LED has a wall-plug efficiency of 0.26, so our value of .23 is not unreasonable.

    %d) The calculated total emitted power
    %e) The calculated supplied electrical power and the wall-plug efficiency
%%%%%%%%%%%%%%%%%%%%%%%%%%%%%%%%%%%%%%%%%%%% 
\subsection{Extra Credit}
As mentioned before, we computed the total power emitted numerically using trapezoid integration which yielded a value of $\SI{4.7}{mW}$. We also analytically computed a value of $ \SI{.03}{W}$. The total power emitted we obtained is 1.5 times larger than our analytically derived value. Some reasone for this discrepancy are presented in the discussion.
\subsection{Resultant Data}

\begin{table}[!htb]
\caption{Calculated Results} \label{tab:calc_results}
\bigskip
\centering
\begin{tabular}{l|l}
\rowcolor[HTML]{C0C0C0} 
\textbf{Parameter} & \textbf{Value} \\ \hline
Raised Cosine Exponent, $n$ & 33.1 \\
Peak Radiation Intensity, $K_0$ & \SI{66.2}{mW/sr} \\
Total Emitted Power, $P_e$ & \SI{16.7}{mW} \\
Supplied Electrical Power, $P_s$ & \SI{12.2}{mW} \\
Wall-Plug Efficiency & 63.235\%
\end{tabular}
\end{table}

%%%%%%%%%%%%%%%%%%%%%%%%%%%%%%%%%%%%%%%%%%%%
\subsection{Polar Plots}
\begin{figure}[H]
    \centering
    \begin{tikzpicture}
    \begin{polaraxis}[
        xticklabels={%
          -$\frac{\pi}{2}$,
          -$\frac{\pi}{4}$,
          0,
          $\frac{\pi}{4}$,
          $\frac{\pi}{2}$,
        },
        xmin=-1.5707963267948966,xmax=1.5707963267948966,
        ymin=0,ymax=1,
        xtick={-1.5707963267948966,-0.7853981633974483,...,1.5707963267948966},
        ytick={0,0.2,...,1},
        ytick pos=left,
        yticklabel style={anchor=east},
        ylabel style={anchor=near ticklabel,rotate=90},
        x coord trafo/.code=\pgfmathparse{#1*360/(2*pi)},
        title= {},
        ylabel={Watts},
        legend entries={Observed Power(W)},
        legend style={anchor=west}
    ]
    \addplot[only marks,blue] table[col sep=comma, x={Angle}, y={wNorm}]
    {dataPlot.csv};
    {dataPlot.csv};
    \end{polaraxis}
    \end{tikzpicture}
    \caption{Observed Power (Normalized)}
    \label{fig:polar power}
\end{figure}

\begin{figure}[H]
    \centering
    \begin{tikzpicture}
    \begin{polaraxis}[
        xticklabels={%
          -$\frac{\pi}{2}$,
          -$\frac{\pi}{4}$,
          0,
          $\frac{\pi}{4}$,
          $\frac{\pi}{2}$,
        },
        xmin=-1.5707963267948966,xmax=1.5707963267948966,
        ymin=0,ymax=1,
        xtick={-1.5707963267948966,-0.7853981633974483,...,1.5707963267948966},
        ytick={0,0.2,...,1},
        ytick pos=left,
        yticklabel style={anchor=east},
        ylabel style={anchor=near ticklabel,rotate=90},
        x coord trafo/.code=\pgfmathparse{#1*360/(2*pi)},
        title= {},
        ylabel={Watts},
        legend entries={Observed Power(W), Fitted Power(W) $\cos ^n$},
        legend style={anchor=west}
    ]
    \addplot[only marks,blue] table[col sep=comma, x={Angle}, y={wNorm}]
    {dataPlot.csv};
    \addplot[no markers,red] table[col sep=comma, x={Angle}, y={BestFitCosine}]
    {dataPlot.csv};
    \end{polaraxis}
    \end{tikzpicture}
    \caption{Observed Power (Normalized) vs. Predicted Power (Lambertian Cosine)}
    \label{fig:polar power}
\end{figure}






    %f) A polar plot (in the manner of Figs. 2 and 3), showing all data points at angle location (plotted as discrete legible points, please)
    %g) A polar plot, over-layed on the polar plot of the data points, showing the raised cosine function (a solid curve) obtained from your curve fit.
    % Polar Plot
